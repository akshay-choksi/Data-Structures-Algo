\documentclass[12pt]{article}

\usepackage[letterpaper]{geometry}
\usepackage{times}
\geometry{top=1.0in, bottom=1.0in, left=0.75in, right=0.75in}
\usepackage{setspace}

\begin{document}
\begin{flushleft}

% PUT YOUR NAME & THE DATE!!!!
YOUR NAME\\
Professor Floryan\\
CS 3100: DSA II\\
MM/DD/2022\\

\begin{center}
Module 3: Divide and Conquer
\end{center}

\setlength{\parindent}{0.5in}

\noindent This homework will ask you to think about Divide and Conquer Algorithms with a focus on
recurrence relations. Please try to follow the guidelines below when writing up your solutions
to these problems:\\
\begin{itemize}
    \item Algorithm descriptions should be about 1 paragraph long. If you are writing more than
one paragraph to describe your algorithm, then it is TOO long.
    \item There should be enough detail that I could implement the algorithm if I wanted to, but we
don’t need to see code. If you want to add pseudo-code for clarity, that would be fine.
    \item For the recurrence relation problems below, make sure to show your work clearly!
\end{itemize}
\bigskip



\newpage
\noindent 1. You are a hacker, trying to gain information on a secret array of size $n$. This array contains
$n$ - $1$ ones and exactly $1$ two; you want to determine the index of the two in the array.\\
\medskip
\noindent Unfortunately, you don’t have access to the array directly; instead, you have access to a function $f(l1, l2)$ that compares the sum of the elements of the secret array whose indices are in $l1$ to those in $l2$. This function returns -1 if the $l1$ sum is larger, 0 if they are equal, and 1 if the sum corresponding to $l2$ is larger.\\
\medskip
\noindent For example, if the array is $a = [1, 1, 1, 2, 1, 1]$ and you call $f([0, 2, 4], [1, 3, 5])$ then the re-turn value is 1 because $a[0] + a[2] + a[4] = 3 < 4 = a[1] + a[3] + a[5]$. Design an algorithm to find the index of the 2 in the array using the least number of calls to $f()$. Then, answer the following questions:
\begin{itemize}
    \item Describe your algorithm clearly (in a paragraph or so)
    \item Give the recurrence relation for the runtime of your algorithm. Make sure to write this
in terms of $f_r$, which we will use to represent the runtime of $f()$.
    \item Suppose you discover that $f()$ runs in $\Theta(max(|l1|, |l2|))$, what is the overall runtime of
your algorithm in this case?
\end{itemize}
\hrulefill
\bigskip

%answer1











\newpage
\noindent Directly solve, by unrolling the recurrence, the following relation to find its exact solution. Make sure to show your work.\\
\medskip
2. $T(n)=T(n-1)+n$ \\
\medskip
\noindent\hrulefill
\bigskip

%answer2










\newpage
\noindent Use induction to show bounds on the following recurrence relation.\\
\medskip
3. Show that $T (n) = 4T(\frac{n}{3}) + n \in O(n^{log_3(4)})$. You’ll need to subtract off a lower-order term to make the induction work here.\\
\medskip
\noindent\hrulefill
\bigskip

%answer3









\newpage
 \noindent Use the master theorem to solve the following recurrence relations. State which case of the
theorem you are using and why.
\begin{enumerate}\addtocounter{enumi}{3}
    \item $T (n) = 2T (\frac{n}{4} ) + 1$
    \item $T (n) = 2T (\frac{n}{4} ) + \sqrt{n}$
    \item $T (n) = 2T (\frac{n}{4} ) + n$
    \item $T (n) = 2T (\frac{n}{4} ) + n^2$
\end{enumerate}
\noindent\hrulefill
\bigskip

%answer4










\end{flushleft}
\end{document}
\}